\chapter{Introducción}

La ciberseguridad es un derivado de la seguridad TI. Involucra todos los procesos y técnicas utilizadas para proteger 
la información digital alojada en computadoras, servidores y redes del acceso no autorizado, ataques o su destrucción.\\

A diferencia de la seguridad TI, que protege tanto la información física como digital, la ciberseguridad 
solo se enfoca en proteger la información digital.\\

En los últimos tiempos el desarrollo de software ha acortado los tiempos  desarrollo, pasando del ciclo clásico de desarrollo 
en cascada con tiempos de desarrollo de meses a un ciclo de desarrollo ágil con un tiempo de desarrollo de semanas. Este acortamiento 
de los tiempos de desarrollo ha llevado apareados cambios en todo los procesos involucrados en el proceso de desarrollo del 
software incluidos los procesos de pruebas de seguridad.\\

Además, el acortamiento de los tiempos de desarrollo también ha provocado la necesidad de una mayor conexión 
ente los distintos miembros involucrados en los distintos procesos de desarrollo del software.\\ 

Es por ello por lo que en el presente PFG trato de describir un proceso de pentesting que ayude a pentester a reducir 
el tiempo necesario en el desarrollo de la pruebas de seguridad necesarias para poder desplegar un proyecto en producción 
en el menor tiempo posible.\\ 

Además de generar mecanismos de comunicación con el equipo de desarrollo enfocados en resolver los problemas que detecten 
dichas pruebas de seguridad en el menor tiempo posible.\\

\newpage
\section{Motivación}

En el tiempo que llevo involucrado en proyectos de la ciberseguridad he visto muchas herramientas y metodologías que abordan 
el proceso de pruebas de seguridad o pruebas de pentesting.\\

Normalmente, el proceso de pentesting es un proceso manual y muy dependiente de la persona que realice las mismas 
y muchas veces los reportes presentado solo se dedican a señalar el problema y no suelen enfocarse a la solución de los problemas.\\

Es por ello por lo que el presente proyecto fin de grado quería plasmar un proceso de pentesting que sirva de punto de partida
para la realización de pruebas de seguridad al pentester y que se encuentre alineado con el ciclo de desarrollo software
y enfocado a la solución de los problemas que se detecten en la evaluación de un software.\\

\section{Objetivos}
Durante la realización de este TFG se pretende realizar una breve introducción al proceso de pentesting, la fase que involucran 
dicho proceso, las herramientas que se pueden utilizar para llevar a cabo dicho proceso, así como la descripción de los fallos más
comunes que se suelen encontrar en las evaluaciones y detallar los documentos que se deben generar enfocados a tener
una comunicación con el equipo de desarrollo encaminados a la solución de los defectos 
y a la mejora de la calidad del software evaluado.\\

Para poder lograr lo que se ha indicado en el párrafo anterior, se plantean los siguientes objetivos:

\begin{itemize}
    \item Detalle del proceso de pentesting y sus fases.
    \item Detallar los defectos de seguridad más comunes que se suelen encontrar en la ejecución de pruebas de seguridad o pentesting.
    \item Detallar la realización de los análisis de código estático y dinámicos de aplicaciones que hagan uso de las tecnologías más comunes en la web.
    \item Desarrollo de una pequeña herramienta que facilite la realización de los reportes de análisis 
    estáticos de código enfocados a la comunicación con el equipo de desarrollo de cara a la solución de los defectos 
    encontrados y a la mejora de a calidad del software analizado. 
    \item Ejecución del proceso de pentesting sobre aplicaciones que usen las tecnologías de desarrollo más comunes 
    en el desarrollo de aplicaciones web.
\end{itemize}

\section{Metodología y Competencias}

Para la realización del PFG haremos uso de la 
metodología \href{https://owasp.org/www-pdf-archive/OWASP_Application_Security_Verification_Standard_4.0-en.pdf}{OWASP Application Security Verification Standard (ASVS) 4.0} \cite{web2} que 
proporciona una base para probar los controles técnicos de seguridad de las aplicaciones web.\\ 

Las competencias que se tienen que seguir para la correcta realización del proyecto son:

\textbf{[CO1]} Capacidad para diseñar, desarrollar, seleccionar y evaluar aplicaciones y sistemas informáticos, asegurando 
su fiabilidad, seguridad y calidad, conforme a principios éticos y a la legislación y normativa vigente.\\

\textbf{[IC6]} Capacidad para comprender, aplicar y gestionar la garantía y seguridad de los sistemas informáticos.\\

\textbf{[TI7]} Capacidad para comprender, aplicar y gestionar la garantía y seguridad de los sistemas informáticos.\\

\section{Estructura del proyecto}

El presente PFG está estructurado en cinco capítulos.\\

En el capítulo primero, se expone la introducción, los objetivos, la metodología, la estructura, etc.\\

En el capítulo dos estará enfocado al análisis del estado del arte, donde expondremos en que consiste el proceso de pentesting y 
sus fases. También se detallarán los defectos de seguridad más comunes encontrados en los procesos de pentesting 
además de describir a nivel teórico los procesos de análisis estático y dinámico de código.\\

El capítulo tres estará enfocado al diseño de la solución técnica para llevar a cabo el proceso de pentesting.\\ 

En capítulo cuatro se hará uso del proceso definido en el capitulo tres sobre aplicaciones desarrolladas con las tecnologías 
más comunes en el desarrollo de aplicaciones web.\\

En el capítulo seis se expondrán las conclusiones del proyecto, así como posibles trabajos futuros sobre este tema.
También se incluye un anexo donde se detallarán los informes generados en la realización del proceso de pentesting.\\

\newpage