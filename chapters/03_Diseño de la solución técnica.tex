
\section{Métodologia de pruebas} 

Como metodología de pruebas para el proceso haremos uso 
de \href{https://owasp.org/www-pdf-archive/OWASP_Application_Security_Verification_Standard_4.0-en.pdf}{OWASP Application Security Verification Standard (ASVS) 4.0} 
que proporciona una base para probar los controles técnicos de seguridad de las aplicaciones web. El proyecto clasifica
 los distintos controles en tres niveles. En este caso cubriremos todos los controles incluidos en el \textbf{nivel 2}.

Para abordar el proceso de pentesting los dividiremos en varias fases que detallaremos a continuación, así como los documentos a generar y herramientas necesarias para cada fase.
Dentro de las fases definidas en apartado 2.1.2 para la ejecución del proceso de pentesting ejecutaremos todas las fases menos la de explotación y Postexplotación.

\section{Infraestructura de pruebas} 
Como entorno de pruebas para la ejecución de los análisis de código; haremos uso de una máquina física y de un contenedor 
de Docker con la siguientes características y herramientas instaladas en cada una de ellas:

Para levantar el contenedor podemos hacer uso de dockercompose incluido en la carpeta \textbf{"entornoPrueba"} dentro de 
las \href{https://github.com/M0l1n3ta/PFG/tree/master}{fuentes de proyecto}