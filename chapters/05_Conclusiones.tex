\section{Conclusiones} 
Una vez finalizado el desarrollo del trabajo creo que se ha conseguido detallar un proceso de pentesting que puede ser 
utilizado por cualquiera para obtener los defectos de seguridad en un proceso reproducible.

Se han detallado herramientas que pueden ser utilizadas, así como los documentos a generar de cara a una comunicación con 
el equipo de desarrollo enfocada a la resolución de los defectos encontrados en el software analizado.

Se ha desarrollado una utilidad que permite la extracción de la información de los análisis estáticos de código 
para incluirlos en reportes basados en una plantilla predefinida.

Se ha aportado un entorno de pruebas reproducible para la ejecución de análisis de código, tanto estático como dinámicos.

Se ha aplicado dicho proceso sobre distintas tecnologías generando los documentos requeridos por el proceso de pentesting 
para la obtención de los defectos presentes en dichas aplicaciones. Además, se han analizado dichos defectos 
incluyendo en los distintos documentos los planes para remediar dicho defectos para cada aplicación.

En mi opinión creo que se ha cumplido el objetivo general que tenía en mente que era detallar todo el proceso de pentesting
para poder ejecutarse por parte de un pentester de forma que proporcione una base y unos documentos 
específicos que pueden servir de partida para la detección de los defectos de seguridad, o defectos en general 
presentes en cualquier código. Como se ha visto en los casos de prueba, dependerá en gran medida del conocimiento 
sobre la aplicación y el diseño de un plan de pruebas en constante evolución que puedan llegar a detectar el 
mayor número de defectos posibles presentes en un determinado código.

\newpage
\section{Trabajo futuro} 

Uno paso lógico, para continuar en la línea del proyecto, sería integrar en proceso descrito en el presente PFG en una \gls{pipeline}
que permitiese la ejecución del proceso de forma automática en los procesos de desarrollo e integración continuos.

Algunas de las tecnologías donde se podría crear dicha \gls{pipeline} podrían ser:

\begin{itemize}
    \item Azure Devops
    \item Jenkin 
    \item GitHub Actions
\end{itemize}

Para lograr integrar el proceso dentro de una \gls{pipeline} habría que modificar la herramienta de creación de reportes para que 
aceptase parámetros en línea de comandos para permitir la integración de la herramienta en dicho proceso.

Uno de los trabajos  a mejorar, del presente PFG, son lo planes de prueba de forma que detecten un mayor número de los defectos 
presentes en cada una de las aplicaciones de prueba.

Añadir más tecnologías de desarrollo web como puede ser el caso de .Net Core, que si bien en un principio opté por incluir tuve 
que descartar puesto que no encontré aplicaciones que tuviesen implementados todos los defectos 
del OWASP top 10 de forma similar a la aplicaciones escogidas para este PFG.
