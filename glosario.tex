
\newglossaryentry{ssdlc}{name=SSDLC, description={Del inglés "Secure Software Development Life Cycle". El Software Development 
Life Cycle (SDLC) es un proceso de desarrollo estructurado enfocado en la producción de sofware de calidad, con el menor costo 
y el period más corto posible de tiempo. 
Un proceso seguro de SDLC, además añade procesos adicionales, encaminados a
mejorar la calidad del sofware, tales como pruebas de penetración, revisiones de código o análisis de dependencias.
}}

\newglossaryentry{owasp}{name=OWASP ,description={
    El Open Web Application Security Project (OWASP) es una comunidad mundial libre y abierta enfocado en mejorar 
    la seguridad del desarroollo de software 
}}
\newglossaryentry{asvs}{name=ASVS ,description={
    El proyecto OWASP Appplication Security Verification Standard (ASVS) 
    proporciona una base para realizar los controles de seguridad técnicos en aplicaciones 
    web, admás también proporciona un listado de requisitos a cumplir para un desarrollo seguro. 
}}

\newglossaryentry{exploit}{name=exploit ,description={
    Término inglés que hace referencia a una secuencia de comandos 
    utilizados para, aprovechándose de un fallo o vulnerabilidad 
    en un sistema, provocar un comportamiento no deseado o imprevisto. 
}}

\newglossaryentry{sca}{name=SCA ,description={
    Del inglés "Static Code Analysys", término que hace referencia a las pruebas de análisis estático de código
}}


\newglossaryentry{sast}{name=SAST ,description={
    Del inglés "Static Application Security Testing", término que hace referencia a las pruebas de análisis estático de código
}}

\newglossaryentry{dast}{name=DAST ,description={
    Del inglés "Dynamic Application Security Testing", término que hace referencia a las pruebas de análisis dinámicas de código.
}}

\newglossaryentry{cd}{name=CD ,description={
    Del inglés "Continous Deployement", término que hace referencia a la implantación de procesoa automáticos de desplige de las aplicaciones en el ciclo de vida del desarrolo de software.
}}

\newglossaryentry{ci}{name=CI ,description={
    Del inglés \emph{"Continous Integración"}, término que hace referencia a la implantación de procesos automáticos de compilación y revison del codigo fuente en el ciclo de vida del desarrollo software.
}}


\newglossaryentry{sut}{name=SUT ,description={
    Del inglés \emph{"Sytem Under Test"}, término que hace referencia a la aplicación o sistema sobre el cual se ejecutarán las pruebas.
}}


\newglossaryentry{cve}{name=CVE ,description={
    Del inglés \emph{“Common Vulnerabilities and Exposure” (CVE)}, es una lista de información registrada sobre vulnerabilidades de seguridad conocidas, en la que cada referencia 
    tiene un número de identificación CVE-ID.
    Está definido y es mantenido por The MITRE Corporation con fondos de la \href{https://es.wikipedia.org/w/index.php?title=National_Cyber_Security_Division&action=edit&redlink=1}{“National Cyber Security Division”} del gobierno de los Estados Unidos de América.
    La información y nomenclatura de esta lista son usadas en la National Vulnerability Database \href{https://nvd.nist.gov/}{(NVD)}, el repositorio de los Estados Unidos de América de información sobre vulnerabilidades.

}}